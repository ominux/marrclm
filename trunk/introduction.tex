\section{Introduction} \label{sec:intro}

Among all memory characteristics, the so-called multibit
operation or multilevel cell (MLC), which exploits the
layout area of a memory device to realize more than one
bit of digital data per cell, is a desirable capability for highdensity
memory application. RRAM modulates the resistance
states to realize the MLC operation. As shown in
Fig. 9, the LRS resistance can be changed by the set current
compliance possibly due to the modulation of the
diameter or number of CFs, while the HRS resistance can
be controlled by the reset stop voltage possibly due to the
modulation of the ruptured CF length [37]. In addition, for
LRS, the current is clamped by the measurement instrument
for the 1R device or by the gate voltage of the transistor
for the 1T1R device. Therefore, minimizing the
overshoot is crucial for achieving multilevel states in LRS.
For a memory array, the current clamping must be provided
by on-chip circuits. Most RRAM material systems,
such as CuOx~\cite{CACTI51} [??], TiOx [??], HfOx [??], WOx [??], and
TaOx [??], were reported to be capable of MLC operation.
The largest number of resistance levels reported so far are
five levels without verification for the HfOx memory [??]
and eight levels with verification for the WOx memory
[??]. Recently, Yu et al. [??] proposed and verified two
equivalent pulse programming schemes to achieve the
multilevel resistance values owing to the exponential
voltage�Ctime relationship: one is to exponentially increase
the programming pulse width; the other is to linearly
increase the programming pulse amplitude. Although both
schemes were effective for achieving the target resistance,
the transient current response measurements suggest the
second scheme consumes considerably less energy for the
programming.

For MLC operation, first, enough resistance window
between any two states and the uniformity of each resistance state are required. Second, the cycling endurance
for each state is also an important criterion.

