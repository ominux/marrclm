\begin{abstract}
Updated Sep 9th 3:25AM PST. Resistive Random Access Memory (ReRAM) recently has been a promising candidate for next-generational memory technology when there is a foreseeable growing demand for large capacity memory in future computer architecture. The capability of storing multiple bits in a single ReRAM cell further improves the memory density and reduces cost per bit significantly. Multi-step write-and-verify scheme is required in multi-level cell (MLC) ReRAM programming due to the existence of all sorts of variations. The traditional programming approaches have the drawbacks of either long average write latency or possibility of retention failure in some programmed states. To the best of our knowledge, we identify the reliability issue in MLC ReRAM application for the first time. In this work, we proposed two programming schemes targeted at reducing write latency/energy and improving the memory reliability separately. Hardware designer can adopt either set of programming approach based on the requirement of memory sub-system specifications in the memory hierarchies.
\end{abstract} 